\section{Задание на практику}
	Произвести сравнительный анализ вариантов реализаций модели с управлением системы (группы лифтов).

	В ходе практики должны быть освоены компетенции:
		\begin{itemize}
			\item способность проектировать вспомогательные и специализированные языки программирования и языки представления данных;
			\item способность анализировать профессиональную информацию, выделять в ней главное, структурировать, оформлять и представлять в виде аналитических обзоров с обоснованными выводами и рекомендациями;
			\item способность проектировать системы с параллельной обработкой данных и высокопроизводительные системы, и их компоненты.
		\end{itemize}

\newpage
\section{Введение}
	На сегодняшний день не существует оптимального алгоритма управления системой группой лифтов, так как каждый из них имеет ряд недостатков.

	Например, в основе классического алгоритма заложен принцип парного и группового управления, который в свою очередь базируется на собирательном управлении,
		то есть вызовы должны регистрироваться и выполняться в соответствии с ограниченным числом условий, которые должны быть учтены при проектировании
		системы управления. Поскольку ограничено количество условий, по которым система управления распределяет вызовы между кабинами
		и свободные кабины по высоте здания, то любая система не во всех ситуациях действует наилучшим образом.

	Таким образом, актуальной является задача разработки алгоритма, в основе которого заложен принцип
	интеллектуального сервера управления лифтовыми группами с целью реализации более сложных алгоритмов, так же это позволит внедрять большее число
		дополнительных функций и связывать между собой согласованную работу лифтов наряду с обеспечением комфортной доставки пассажиров.
