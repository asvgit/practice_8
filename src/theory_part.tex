\stepcounter{mysection}\section{\arabic{mysection} Обзор программных реализаций}

	Целью выпускной квалификационной работы является реализация интеллектуальной системы автоматического управления.
		В качестве объекта исследования была выбрана задача о лифтах, то есть применение логического вывода и
		порождения гипотез в задаче управления и исследования свойств системы.

	Для решения поставленной задачи является необходимостью разработать модель и произвести с её
		помощью исследование. Входе проведения исследования было реализовано три варианта модели.
		В каждом из вариантов процесс обработки запросов различен. Поэтому будет весьма полезным испытать
		варианты реализации в одинаковых условиях и произвести сравнительный анализ.

\subsection{Реализация на языке Python}
	Первый вариант является простейшим, кабины рассматриваются в системе как общедоступный ресурс,
		а при появлении человека происходит запрос на данный ресурс, если же все ресурсы в момент вызова
		являются занятыми, то происходит ожидание человеком возможности занять этот ресурс.

	Например, есть система из k = 2 кабин, способных перемещаться по n =
		5 этажам. Пусть кабины находятся на 1-м и 3-м этажах, первая пуста и находится
		в покои, а второй предстоит остановка на 4-м этаже. Поступает
		вызов с пятого этажа, и вторая кабина получается ближайшей, но вызвана будет первая.
		Так как она является в покои, несмотря на тот факт, что вторая кабины доберётся быстрее.
		Если в данной ситуации появится ещё один человек, ему придётся ждать освобождения ресурса.

\subsection{Реализация на языке Prolog}
	Простейшим алгоритмом принятия решения является поиск ближайшей
		кабины к месту вызова. Однако, термин «ближайшая» требует уточнения и рассмотрения примера.

	Допустим, есть система из k = 2 кабин, способных перемещаться по n =
		5 этажам. Пусть кабины находятся на 1-м и 2-м этажах, первая пуста и находится
		в покои, а второй предстоят остановки на 3-м и 4-м этажах. Поступает
		вызов с пятого этажа, и первая кабина получается ближайшей, так как её требуется
		4 такта, а второй кабине требуется 5 тактов. Получается дистанция –
		это количество тактов, которое необходимо кабине, чтобы добраться до этажа,
		выполняя уже сформированный маршрут [1].
\subsection{Реализация на языке Prolog с построением гипотиз}
	Последний подход основывается на исключении худших альтернатив на основе логического вывода.
		И если после сокращения допустимых альтернатив их останется несколько, то выбор может быть случайным
		или основываться на каких-либо критериях. В этом и предыдущих подходах
		одним из основных критериев является средняя длительность ожиданий.

	Основными объектами в данной модели являются кабина Cab и человек Man.
		В момент времени t кабина имеет вид Cab(i, e, S, t), где i – идентификатор кабины, e – этаж,
		а S - маршрут кабины, список этажей. Человек имеет вид Man(e, d, τ, t), где e – этаж,
		d – целевой этаж, который добавляется в маршрут S в момент входа человека в кабину и d ≠ e,
		τ – длительность ожидания человеком кабины.
		Дистанцией же будет Dist(e, S, i, t, α), где α – это дистанция от кабины i
		на этаже е с маршрутом S, где произошёл вызов. И связь i кабины с вызовом с e этажа Conn(i, e).

	В каждый момент времени $t_0$ принятия решения формула F имеет вид:\\

	$ \exists A(t_0) \begin{cases} \forall T(t)\exists T(t'), N(t, t'), \\ \Phi \\ \Psi \end{cases} $\\

	$A(t_0)$ -- коньюнкт, описывающий состояние системы в момент времени $t_0$. Если $A(t_0)$ содержит Man,
		то появление человека необходимо связать с вызовом определённой кабины.
		И группа формул \Phi порождает все варианты связи и имитирует движение кабин совместно с
		формулой времени $\forall T(t)\exists T(t'), N(t, t')$ для некоторого количества тактов.
		А за счёт формул \Psi происходит фильтрация некоторых вариантов.

	Оставшиеся варианты оцениваются оцениваются и выбирается один из самых наилучших.
